\documentclass[oneside,]{memoir}
\usepackage{amssymb,amsmath}
\usepackage{ifxetex,ifluatex}
\ifxetex
  \usepackage{fontspec,xltxtra,xunicode}
  \defaultfontfeatures{Mapping=tex-text,Scale=MatchLowercase}
  \newcommand{\euro}{€}
\else
  \ifluatex
    \usepackage{fontspec}
    \defaultfontfeatures{Mapping=tex-text,Scale=MatchLowercase}
    \newcommand{\euro}{€}
  \else
    \usepackage[utf8]{inputenc}
  \fi
\fi
% Redefine labelwidth for lists; otherwise, the enumerate package will cause
% markers to extend beyond the left margin.
\makeatletter\AtBeginDocument{%
  \renewcommand{\@listi}
    {\setlength{\labelwidth}{4em}}
}\makeatother
\usepackage{enumerate}
\usepackage{graphicx}
% We will generate all images so they have a width \maxwidth. This means
% that they will get their normal width if they fit onto the page, but
% are scaled down if they would overflow the margins.
\makeatletter
\def\maxwidth{\ifdim\Gin@nat@width>\linewidth\linewidth
\else\Gin@nat@width\fi}
\makeatother
\let\Oldincludegraphics\includegraphics
\renewcommand{\includegraphics}[1]{\Oldincludegraphics[width=\maxwidth]{#1}}
\ifxetex
  \usepackage[setpagesize=false, % page size defined by xetex
              unicode=false, % unicode breaks when used with xetex
              xetex,
              bookmarks=true,
              pdfauthor={},
              pdftitle={},
              colorlinks=true,
              linkcolor=blue]{hyperref}
\else
  \usepackage[unicode=true,
              bookmarks=true,
              pdfauthor={},
              pdftitle={},
              colorlinks=true,
              linkcolor=blue]{hyperref}
\fi
\hypersetup{breaklinks=true, pdfborder={0 0 0}}
\setlength{\parindent}{0pt}
\setlength{\parskip}{6pt plus 2pt minus 1pt}
\setlength{\emergencystretch}{3em}  % prevent overfull lines
\usepackage{fourier} % or what ever
\usepackage[scaled=.92]{helvet}%. Sans serif - Helvetica
\usepackage{color,calc}
\newsavebox{\ChpNumBox}
\definecolor{ChapBlue}{rgb}{0.00,0.65,0.65}
\definecolor{DarkBlue}{rgb}{0.00,0.20,0.65}
\definecolor{Black}{rgb}{0.00,0.00,0.00}
\makeatletter
\newcommand*{\thickhrulefill}{%
\leavevmode\leaders\hrule height 1\p@ \hfill \kern \z@}
\newcommand*\BuildChpNum[2]{%
\begin{tabular}[t]{@{}c@{}}
\makebox[0pt][c]{#1\strut} \\[.5ex]
\colorbox{ChapBlue}{%
\rule[-10em]{0pt}{0pt}%
\rule{1ex}{0pt}\color{black}#2\strut
\rule{1ex}{0pt}}%
\end{tabular}}
\makechapterstyle{BlueBox}{%
\renewcommand{\chapnamefont}{\large\scshape}
\renewcommand{\chapnumfont}{\Huge\bfseries}
\renewcommand{\chaptitlefont}{\raggedright\Huge\sffamily\bfseries}
\setlength{\beforechapskip}{20pt}
\setlength{\midchapskip}{26pt}
\setlength{\afterchapskip}{40pt}
\renewcommand{\printchaptername}{}
\renewcommand{\chapternamenum}{}
\renewcommand{\printchapternum}{%
\sbox{\ChpNumBox}{%
\BuildChpNum{\chapnamefont\@chapapp}%
{\chapnumfont\thechapter}}}
\renewcommand{\printchapternonum}{%
\sbox{\ChpNumBox}{%
\BuildChpNum{\chapnamefont\vphantom{\@chapapp}}%
{\chapnumfont\hphantom{\thechapter}}}}
\renewcommand{\afterchapternum}{}
\renewcommand{\printchaptertitle}[1]{%
\usebox{\ChpNumBox}\hfill
\parbox[t]{\hsize-\wd\ChpNumBox-1em}{%
\vspace{\midchapskip}%
\thickhrulefill\par
\chaptitlefont ##1\par}}%
}
\chapterstyle{BlueBox}
%\chapterstyle{dash}
%%% Creating an initial of the very first character of the content
\usepackage{lettrine}
\newcommand{\initial}[1]{
	\lettrine[lines=3,lhang=0.3,nindent=0em]{
		\color{ChapBlue}
		{\textsf{#1}}
	}{}
}

%%% Title, author and date metadata
\usepackage{titling} % For custom titles
\newcommand{\HorRule}{
	\color{DarkBlue} % Creating a horizontal rule
	\rule{\linewidth}{1pt}
}
\pretitle{
	\vspace{-30pt} \begin{flushleft} \HorRule 
	\fontsize{50}{50} \usefont{OT1}{phv}{b}{n} \color{ChapBlue} \selectfont 
}
\posttitle{\par\end{flushleft}\vskip 0.5em}

\preauthor{\begin{flushleft}\large \lineskip 0.5em \usefont{OT1}{phv}{b}{sl} \color{Black}}
\postauthor{\footnotesize \usefont{OT1}{phv}{m}{sl} \color{Black} 
	,\qquad % One could have the institution of author right here
	\par\end{flushleft}\HorRule
}
\newcommand{\modreq}{!}
\newcommand{\modquery}{?}
\newcommand{\modtrue}{\top}
\newcommand{\modfalse}{\bot}
\newcommand{\moderr}{\epsilon}
\newcommand{\modaux}{a}


\begin{document}

\chapter{pandocgen}

This is a generation framework for yielding PDF and HTML from good old
Markdown files. It uses the eminent
\href{http://johnmacfarlane.net/pandoc/}{Pandoc tool}, so the Markdown
files can use Pandoc extensions to provide a slicker output, including
mathematical expressions. For the pure Markdown syntax and its
semantics, there is a good introduction on
\href{http://daringfireball.net/projects/markdown/syntax/}{Daring
Fireball}.

\textbf{NOTE}: for some reason, the creator of Pandoc, John MacFarlane,
has not named the Pandoc Markdown extension language. I sometimes refer
to it as \textbf{PandocMarkdown}.

\textbf{NOTE}: Pandoc is capable of translating between a host of
formats, but this \textbf{pandocgen} project focuses on
(Pandoc-)Markdown input. See the graph at the bottom of this document
for all the various conversion options of Pandoc; it is quite
mind-blowing. The image is from the Pandoc site and is copyrighted by
John MacFarlane.

There are some current issues relative links to resources. See the
Issues section below.

\section{Very, Very Quick Intro\ldots{}}

\begin{enumerate}[1.]
\item
  Add this project as a submodule

\begin{verbatim}
git submodule add git@github.com:davber/pandocgen.git
\end{verbatim}
\item
  Create some beautiful Markdown file, \texttt{MyCoolDoc.md}
\item
  Create a Makefile like this:

\begin{verbatim}
BASE=MyCoolDoc
include pandocgen/Makefile
\end{verbatim}
\item
  Make space for the generated files:

\begin{verbatim}
mkdir gen
\end{verbatim}
\item
  Create the PDF and HTML files:

\begin{verbatim}
make all
\end{verbatim}
\end{enumerate}
That is it! Now you have a PDF and HTML version of your Markdown
document.

For a quick generation, you can actually generate this README file ---
people always enjoy others eating their own dog food (or actually others
eating pet food in general\ldots{}) --- by

\begin{verbatim}
make --file sample.mk
\end{verbatim}
where \texttt{sample.mk} is this short make file

\begin{verbatim}
BASE=README
RES_OUT=rez/diagram.png
include Makefile
\end{verbatim}
This will generate output files in the \texttt{gen} directory. They are
also uploaded to this Wiki, as a \href{./gen/README.pdf}{PDF file} and
\href{./gen/README.html}{HTML file}.

\section{Dependencies}

There are some dependencies, though:

\begin{enumerate}[1.]
\item
  A Gnu \textbf{make} command, preferably version 3.80 or later. On most
  decent machines, this is already installed.
\item
  LaTex, such as \href{http://www.tug.org/texlive/}{TeX Live}.
\item
  \href{http://johnmacfarlane.net/pandoc/}{Pandoc} --- the tool actually
  performing the generation
\end{enumerate}
\section{Using It\ldots{}}

To use this framework, there are two paths:

\begin{enumerate}[1.]
\item
  Include the provided Makefile in your own Makefile, after a section
  where you specify a few custom parameters, as defined in the Custom
  Parameters section.
\item
  Set the custom parameters as environment variables and use the
  provided Makefile as is.
\end{enumerate}
\section{Building Targets}

All target versions of the Markdown document are generated in a
\texttt{gen} directory relative the current working directory. As a
convenience, this project contains such a directory in case you are
running \texttt{make} from there.

Each of the target formats has a corresponding make target, so you can
issue one of:

\begin{verbatim}
make pdf
make html
make tex
\end{verbatim}
There is also a universal target, which builds all formats:

\begin{verbatim}
make all
\end{verbatim}
\section{Customer Parameters}

These are the parameters you can set -- either via a initial section in
an embedding Makefile or as environment variables:

\begin{itemize}
\item
  \texttt{BASE} - that is the name of the Markdown document,
  \textbf{without} suffix, such as \texttt{MyCoolDoc}, which will then
  generate \texttt{MyCoolDoc.pdf}, \texttt{MyCoolDoc.html} and
  \texttt{MyCoolDoc.tex} in the \texttt{gen} directory.
  \textbf{MANDATORY}
\item
  \texttt{RES\_IN} - the input files for resources, such as images,
  needed by the document. This often includes Graphviz Dot files or
  other input formats for PDF- and PNG-based images. \textbf{OPTIONAL}
\item
  \texttt{RES\_OUT} - the corresponding generated resource files, which
  are often PDF and PNG files. \textbf{OPTIONAL}.
\item
  \texttt{RES\_GEN} - the full command to generate the \texttt{RES\_OUT}
  files from the \texttt{RES\_IN} files. \textbf{OPTIONAL}
\end{itemize}
\textbf{NOTE}: so there is only \textbf{one} mandatory parameter to set,
and that is \texttt{BASE}.

\textbf{NOTE}: the default behavior, described above, actually allows
you to include resources in an input-ready format, such as raw PNG and
PDF files, by merely setting the \texttt{RES\_OUT} to those files and
let the other two resource-related parameters be. That will translate
into a no-op for that make step.

\section{Helper Files}

The helper files reside in the \texttt{input} directory.

The helper files are:

\begin{itemize}
\item
  \texttt{my-template.latex} - this is the main template for LaTeX
  generation and, indirectly, for PDF generation. It uses some
  parameters that can be set from command line --- and is actually set
  by the provided Makefile --- such as \texttt{docuemntclass}, which the
  Makefile sets to \texttt{memoir}. See the Makefile for some of those
  parameters used.
\item
  \texttt{mytitle.tex} - this is the template for the title page, for
  LaTeX (and PDF\ldots{})
\item
  \texttt{mychapter.tex} - specifies the look of chapter headings for
  LaTeX (and PDF\ldots{})
\item
  \texttt{macros.tex} - some TeX macros. \textbf{NOTE} these macros are
  actually expanded by Pandoc itself in the case of non-TeX-based
  generation --- such as HTML --- so one can have shortcuts or other
  macros even for HTML.
\end{itemize}
\section{Issues}

The way to link to relative resources or to include images in general is
confusing at best in the Markdown world. This README file, for instance
is trying hard to be Markdown, GitHub Markdown and Pandoc Markdown
friendly, so it can render well on GitHub while generating PDF and HTML
that will reference relative resources properly. GitHub's Gollum can
sometimes change its behavior suddenly as how to interpret simple
relative paths, and make them not relative the project --- which they
\textbf{should} be --- but rather relative the top directory of the
GitHub user! So, in order for the diagram to show up properly, I had to
use both the GitHub Markdown syntax of

\begin{verbatim}
[[rez/diagram.png]]
\end{verbatim}
and the syntax expected by Pandoc to generate proper PDF

\begin{verbatim}
![Pandoc Format Conversions](rez/diagram.png)
\end{verbatim}
Also be aware of the intricacies of opening HTML files from the
\texttt{gen} directory vs opening them from the top directory. For
instance, the links and images are relative the top directory of this
project, so in order for relative links to work, you must copy the
\texttt{gen/README.html} to the top directory.

\section{The Completely Connected World Of Pandoc}

Can you count the number of translations possible? \ldots{}

{[}{[}rez/diagram.png{]}{]}

\begin{figure}[htbp]
\centering
\includegraphics{rez/diagram.png}
\caption{Pandoc Format Conversions}
\end{figure}

Again: Image copyright John MacFarlane

\end{document}
